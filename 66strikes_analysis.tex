% Options for packages loaded elsewhere
\PassOptionsToPackage{unicode}{hyperref}
\PassOptionsToPackage{hyphens}{url}
%
\documentclass[
]{article}
\usepackage{amsmath,amssymb}
\usepackage{iftex}
\ifPDFTeX
  \usepackage[T1]{fontenc}
  \usepackage[utf8]{inputenc}
  \usepackage{textcomp} % provide euro and other symbols
\else % if luatex or xetex
  \usepackage{unicode-math} % this also loads fontspec
  \defaultfontfeatures{Scale=MatchLowercase}
  \defaultfontfeatures[\rmfamily]{Ligatures=TeX,Scale=1}
\fi
\usepackage{lmodern}
\ifPDFTeX\else
  % xetex/luatex font selection
\fi
% Use upquote if available, for straight quotes in verbatim environments
\IfFileExists{upquote.sty}{\usepackage{upquote}}{}
\IfFileExists{microtype.sty}{% use microtype if available
  \usepackage[]{microtype}
  \UseMicrotypeSet[protrusion]{basicmath} % disable protrusion for tt fonts
}{}
\makeatletter
\@ifundefined{KOMAClassName}{% if non-KOMA class
  \IfFileExists{parskip.sty}{%
    \usepackage{parskip}
  }{% else
    \setlength{\parindent}{0pt}
    \setlength{\parskip}{6pt plus 2pt minus 1pt}}
}{% if KOMA class
  \KOMAoptions{parskip=half}}
\makeatother
\usepackage{xcolor}
\usepackage[margin=1in]{geometry}
\usepackage{longtable,booktabs,array}
\usepackage{calc} % for calculating minipage widths
% Correct order of tables after \paragraph or \subparagraph
\usepackage{etoolbox}
\makeatletter
\patchcmd\longtable{\par}{\if@noskipsec\mbox{}\fi\par}{}{}
\makeatother
% Allow footnotes in longtable head/foot
\IfFileExists{footnotehyper.sty}{\usepackage{footnotehyper}}{\usepackage{footnote}}
\makesavenoteenv{longtable}
\usepackage{graphicx}
\makeatletter
\def\maxwidth{\ifdim\Gin@nat@width>\linewidth\linewidth\else\Gin@nat@width\fi}
\def\maxheight{\ifdim\Gin@nat@height>\textheight\textheight\else\Gin@nat@height\fi}
\makeatother
% Scale images if necessary, so that they will not overflow the page
% margins by default, and it is still possible to overwrite the defaults
% using explicit options in \includegraphics[width, height, ...]{}
\setkeys{Gin}{width=\maxwidth,height=\maxheight,keepaspectratio}
% Set default figure placement to htbp
\makeatletter
\def\fps@figure{htbp}
\makeatother
\setlength{\emergencystretch}{3em} % prevent overfull lines
\providecommand{\tightlist}{%
  \setlength{\itemsep}{0pt}\setlength{\parskip}{0pt}}
\setcounter{secnumdepth}{-\maxdimen} % remove section numbering
\ifLuaTeX
  \usepackage{selnolig}  % disable illegal ligatures
\fi
\usepackage{bookmark}
\IfFileExists{xurl.sty}{\usepackage{xurl}}{} % add URL line breaks if available
\urlstyle{same}
\hypersetup{
  pdftitle={Strikes in Steel Belt US Military Industry, 1966},
  pdfauthor={Alex Trahan},
  hidelinks,
  pdfcreator={LaTeX via pandoc}}

\title{Strikes in Steel Belt US Military Industry, 1966}
\author{Alex Trahan}
\date{2024-10-10}

\begin{document}
\maketitle

From 1962 to 1965, the Labor Relations staff attached to US Army
Mobility Command (MOCOM), based at the Detroit Tank Arsenal in Warren,
Michigan, kept scattered, mostly handwritten records of the few dozen
work stoppages that military contractors were required to report. Then,
in 1966, strike activity inside military contractors plants exploded,
producing enough reports - now standardized through a special form label
DD-1507: Work Stoppage Report - to fill an entire US National Archives
FRC box.

The picture looking out from the DTA Labor Relations staff records
stands in sharp contrast to existing official statistics on labor
militancy in the defense industry.Under ``Ordnance and accessories''
(read: military industry), the Bureau of Labor Statistics (BLS) Work
Stoppage Analysis for the year 1966 counts only thirteen strikes
involving 8680 workers who idled for a total of 62,500 days. The
documents analyzed in this data, which were generated by the Labor
Relations staff at the Detroit Tank Arsenal (DTA) in that same year,
tell a very different story: almost ten times as many strikes in
military industrial plants, involving over five times as many workers,
who cost their employers at least 790,000 worker-days.

These DTA records provide only a partial scan of the whole body of
military industry in the year 1966. DTA Labor Relations staff reporting
on work stoppages focused almost exclusively on contracts overseen by
the staff's parent agency, MOCOM. We rarely glimpse occurrences of labor
unrest affecting contracts overseen by other Army commands, much less
other branches of the US military. That said, MOCOM oversaw contracts
for a number of big-ticket conventional hardware, like the M60 tank, the
M109 self-propelled howitzer, and M110 armored personnel carrier, as
well as many smaller but still important items, like trucks, shop vans,
and the small horse-power engines used as generators to run AC in
tropical Southeast Asia. This scan may be incomplete, but it captures
vital organs of military industry, and its gaps merely point beyond
itself to an even wider discrepancy between BLS data and the actual
reality of shopfloor conflicts brought to boil by the intensification of
US military involvement in the Vietnam War. The total number of strikes,
strikers, and production days lost likely to be much higher than what is
recorded in this document.

The stakes of this discrepancy between BLS data and DTA records are much
deeper than simple accurate account; it matters for our understanding of
how the United States' long, brutal, and industrially intense war in
Southeast Asia shaped political-economic and social relations on its own
soil. We know that the United States turned Southeast Asia into the most
bombed region on the planet, dropping more bombs at a faster rate, than
it used across all theatres of the Second World War. But we do not yet
know enough about how all this materiel was made, who made it, and what
effects its production had on social reality during a disavowed war with
a disavowed home front.

\emph{Nota Bene:} The purpose of this document is to serve as a
heuristic for guiding research questions and in-depth qualitative
investigation. It is neither a formal preview of a finished research
project nor even a prospectus for a regimented quantitative
analysis.does

\newpage

\section{BLS Data for 1966}\label{bls-data-for-1966}

\begin{figure}
\includegraphics[width=1\linewidth]{.} \caption{BLS Work Stoppages Analyis, 1966 (p. 9)}\label{fig:bls}
\end{figure}
\newpage

\section{Summary Statistics DTA Labor Relations Records,
1966}\label{summary-statistics-dta-labor-relations-records-1966}

\begin{longtable}[]{@{}lr@{}}
\caption{Strike events by type}\tabularnewline
\toprule\noalign{}
Type & count \\
\midrule\noalign{}
\endfirsthead
\toprule\noalign{}
Type & count \\
\midrule\noalign{}
\endhead
\bottomrule\noalign{}
\endlastfoot
authorized strike & 75 \\
potential strike (averted) & 7 \\
wildcat strike & 21 \\
\end{longtable}

\begin{longtable}[]{@{}ll@{}}
\caption{Summary stats for Actual Strikes (Authorized +
Wildcat)}\tabularnewline
\toprule\noalign{}
\endfirsthead
\endhead
\bottomrule\noalign{}
\endlastfoot
Total Strikes & 96 \\
Total Strikers & 49,052 \\
Worker-Days Lost & 789,992 \\
\end{longtable}

\begin{longtable}[]{@{}lr@{}}
\caption{Workers on Strike}\tabularnewline
\toprule\noalign{}
\endfirsthead
\endhead
\bottomrule\noalign{}
\endlastfoot
Minimum Strikers & 3 \\
Maximum Strikers & 5200 \\
Mean Strikers & 831 \\
Median Strikers & 320 \\
Standard Deviation & 1108 \\
\end{longtable}

\begin{longtable}[]{@{}lr@{}}
\caption{Strike Length}\tabularnewline
\toprule\noalign{}
\endfirsthead
\endhead
\bottomrule\noalign{}
\endlastfoot
Mean Strike Length & 26 \\
Median Length & 13 \\
Minimum Strike Length & 1 \\
Maximum Strike Length & 183 \\
\end{longtable}

\begin{longtable}[]{@{}lr@{}}
\caption{Work-Days Lost}\tabularnewline
\toprule\noalign{}
\endfirsthead
\endhead
\bottomrule\noalign{}
\endlastfoot
Minimum Worker-Days Lost & 3 \\
Maximum Worker-Days Lost & 78000 \\
Mean Worker-Days Lost & 14107 \\
Median Worker-Days Lost & 8050 \\
Average Percentage of Worker-Time Lost & 7 \\
\end{longtable}

\begin{longtable}[]{@{}r@{}}
\caption{Estimated Percentage of Production Time Lost by Contractors in
Data}\tabularnewline
\toprule\noalign{}
\endfirsthead
\endhead
\bottomrule\noalign{}
\endlastfoot
6.795091 \\
\end{longtable}

How the above value is calculated: (sum of days lost to strike) /
{[}(sum of strikes) * 365){]}*100. This slightly over estimates
percentage of productivity days lost because it double counts a small
number of factories (N= \textasciitilde4) that featured multiple strikes
and assumes that every factory operated 365 days a year.

\section{Strikes and Strikers Over
Time}\label{strikes-and-strikers-over-time}

\begin{longtable}[]{@{}
  >{\raggedright\arraybackslash}p{(\columnwidth - 24\tabcolsep) * \real{0.2055}}
  >{\raggedleft\arraybackslash}p{(\columnwidth - 24\tabcolsep) * \real{0.0685}}
  >{\raggedleft\arraybackslash}p{(\columnwidth - 24\tabcolsep) * \real{0.0685}}
  >{\raggedleft\arraybackslash}p{(\columnwidth - 24\tabcolsep) * \real{0.0685}}
  >{\raggedleft\arraybackslash}p{(\columnwidth - 24\tabcolsep) * \real{0.0685}}
  >{\raggedleft\arraybackslash}p{(\columnwidth - 24\tabcolsep) * \real{0.0685}}
  >{\raggedleft\arraybackslash}p{(\columnwidth - 24\tabcolsep) * \real{0.0685}}
  >{\raggedleft\arraybackslash}p{(\columnwidth - 24\tabcolsep) * \real{0.0685}}
  >{\raggedleft\arraybackslash}p{(\columnwidth - 24\tabcolsep) * \real{0.0548}}
  >{\raggedleft\arraybackslash}p{(\columnwidth - 24\tabcolsep) * \real{0.0685}}
  >{\raggedleft\arraybackslash}p{(\columnwidth - 24\tabcolsep) * \real{0.0685}}
  >{\raggedleft\arraybackslash}p{(\columnwidth - 24\tabcolsep) * \real{0.0685}}
  >{\raggedleft\arraybackslash}p{(\columnwidth - 24\tabcolsep) * \real{0.0548}}@{}}
\caption{Total Strikes and Total Strikers by Month, 1966}\tabularnewline
\toprule\noalign{}
\begin{minipage}[b]{\linewidth}\raggedright
Metric
\end{minipage} & \begin{minipage}[b]{\linewidth}\raggedleft
Jan
\end{minipage} & \begin{minipage}[b]{\linewidth}\raggedleft
Feb
\end{minipage} & \begin{minipage}[b]{\linewidth}\raggedleft
Mar
\end{minipage} & \begin{minipage}[b]{\linewidth}\raggedleft
Apr
\end{minipage} & \begin{minipage}[b]{\linewidth}\raggedleft
May
\end{minipage} & \begin{minipage}[b]{\linewidth}\raggedleft
Jun
\end{minipage} & \begin{minipage}[b]{\linewidth}\raggedleft
Jul
\end{minipage} & \begin{minipage}[b]{\linewidth}\raggedleft
Aug
\end{minipage} & \begin{minipage}[b]{\linewidth}\raggedleft
Sep
\end{minipage} & \begin{minipage}[b]{\linewidth}\raggedleft
Oct
\end{minipage} & \begin{minipage}[b]{\linewidth}\raggedleft
Nov
\end{minipage} & \begin{minipage}[b]{\linewidth}\raggedleft
Dec
\end{minipage} \\
\midrule\noalign{}
\endfirsthead
\toprule\noalign{}
\begin{minipage}[b]{\linewidth}\raggedright
Metric
\end{minipage} & \begin{minipage}[b]{\linewidth}\raggedleft
Jan
\end{minipage} & \begin{minipage}[b]{\linewidth}\raggedleft
Feb
\end{minipage} & \begin{minipage}[b]{\linewidth}\raggedleft
Mar
\end{minipage} & \begin{minipage}[b]{\linewidth}\raggedleft
Apr
\end{minipage} & \begin{minipage}[b]{\linewidth}\raggedleft
May
\end{minipage} & \begin{minipage}[b]{\linewidth}\raggedleft
Jun
\end{minipage} & \begin{minipage}[b]{\linewidth}\raggedleft
Jul
\end{minipage} & \begin{minipage}[b]{\linewidth}\raggedleft
Aug
\end{minipage} & \begin{minipage}[b]{\linewidth}\raggedleft
Sep
\end{minipage} & \begin{minipage}[b]{\linewidth}\raggedleft
Oct
\end{minipage} & \begin{minipage}[b]{\linewidth}\raggedleft
Nov
\end{minipage} & \begin{minipage}[b]{\linewidth}\raggedleft
Dec
\end{minipage} \\
\midrule\noalign{}
\endhead
\bottomrule\noalign{}
\endlastfoot
Total Strikes & 6 & 9 & 10 & 10 & 6 & 12 & 10 & 7 & 8 & 8 & 7 & 3 \\
Total Strikers & 7433 & 8020 & 6574 & 5732 & 3089 & 1468 & 1433 & 623 &
2360 & 2452 & 9808 & 60 \\
\end{longtable}

\begin{verbatim}
## `geom_smooth()` using formula = 'y ~ x'
\end{verbatim}

\includegraphics{66strikes_analysis_files/figure-latex/time_graph-1.pdf}

\begin{verbatim}
## `geom_smooth()` using formula = 'y ~ x'
\end{verbatim}

\includegraphics{66strikes_analysis_files/figure-latex/time_graph-2.pdf}

\newpage

\section{Geography of Military Industry
Strikes}\label{geography-of-military-industry-strikes}

\begin{verbatim}
## PhantomJS not found. You can install it with webshot::install_phantomjs(). If it is installed, please make sure the phantomjs executable can be found via the PATH variable.
\end{verbatim}

\paragraph{The following two maps are interactive when viewed in HTML.
Click on a marker to reveal specific information. These interactive
functions are not available in PDF
format.}\label{the-following-two-maps-are-interactive-when-viewed-in-html.-click-on-a-marker-to-reveal-specific-information.-these-interactive-functions-are-not-available-in-pdf-format.}

\begin{verbatim}
## Warning in validateCoords(lng, lat, funcName): Data contains 5 rows with either
## missing or invalid lat/lon values and will be ignored
\end{verbatim}

\section{What cities are in the
data?}\label{what-cities-are-in-the-data}

\begin{longtable}[]{@{}r@{}}
\caption{Number of 1-Strike Cities}\tabularnewline
\toprule\noalign{}
num\_cities\_once \\
\midrule\noalign{}
\endfirsthead
\toprule\noalign{}
num\_cities\_once \\
\midrule\noalign{}
\endhead
\bottomrule\noalign{}
\endlastfoot
65 \\
\end{longtable}

\begin{longtable}[]{@{}lrr@{}}
\caption{Top 10 Cities by Number of Strikes}\tabularnewline
\toprule\noalign{}
Location & total\_strikes & total\_strikers \\
\midrule\noalign{}
\endfirsthead
\toprule\noalign{}
Location & total\_strikes & total\_strikers \\
\midrule\noalign{}
\endhead
\bottomrule\noalign{}
\endlastfoot
Detroit, Michigan & 5 & 350 \\
Lima, Ohio & 4 & 2029 \\
Chicago, Illinois & 3 & 2413 \\
Dayton, Ohio & 3 & 600 \\
Milwaukee, Wisconsin & 3 & 450 \\
Akron, Ohio & 2 & 1600 \\
Buffalo, New York & 2 & 150 \\
Elyria, Ohio & 2 & 0 \\
Fort Wayne, Indiana & 2 & 1065 \\
San Jose, California & 2 & 1219 \\
\end{longtable}

\begin{longtable}[]{@{}lrr@{}}
\caption{Top 10 Cities by Number of Strikers}\tabularnewline
\toprule\noalign{}
Location & total\_strikes & total\_strikers \\
\midrule\noalign{}
\endfirsthead
\toprule\noalign{}
Location & total\_strikes & total\_strikers \\
\midrule\noalign{}
\endhead
\bottomrule\noalign{}
\endlastfoot
Columbus, Indiana & 1 & 5200 \\
Stratford, Connecticut & 1 & 4000 \\
Newark, Ohio & 1 & 2500 \\
Chicago, Illinois & 3 & 2413 \\
Northlake, Illinois & 1 & 2105 \\
Lima, Ohio & 4 & 2029 \\
Marion, Indiana & 1 & 1839 \\
Akron, Ohio & 2 & 1600 \\
Columbus, Ohio & 1 & 1350 \\
Milwaukee, WIsconsin & 1 & 1250 \\
\end{longtable}

\begin{longtable}[]{@{}
  >{\raggedright\arraybackslash}p{(\columnwidth - 4\tabcolsep) * \real{0.6081}}
  >{\raggedleft\arraybackslash}p{(\columnwidth - 4\tabcolsep) * \real{0.1892}}
  >{\raggedleft\arraybackslash}p{(\columnwidth - 4\tabcolsep) * \real{0.2027}}@{}}
\caption{Summary of Strikes and Employees on Strike by
City}\tabularnewline
\toprule\noalign{}
\begin{minipage}[b]{\linewidth}\raggedright
Location
\end{minipage} & \begin{minipage}[b]{\linewidth}\raggedleft
total\_strikes
\end{minipage} & \begin{minipage}[b]{\linewidth}\raggedleft
total\_strikers
\end{minipage} \\
\midrule\noalign{}
\endfirsthead
\toprule\noalign{}
\begin{minipage}[b]{\linewidth}\raggedright
Location
\end{minipage} & \begin{minipage}[b]{\linewidth}\raggedleft
total\_strikes
\end{minipage} & \begin{minipage}[b]{\linewidth}\raggedleft
total\_strikers
\end{minipage} \\
\midrule\noalign{}
\endhead
\bottomrule\noalign{}
\endlastfoot
Columbus, Indiana & 1 & 5200 \\
Stratford, Connecticut & 1 & 4000 \\
Newark, Ohio & 1 & 2500 \\
Chicago, Illinois & 3 & 2413 \\
Northlake, Illinois & 1 & 2105 \\
Lima, Ohio & 4 & 2029 \\
Marion, Indiana & 1 & 1839 \\
Akron, Ohio & 2 & 1600 \\
Columbus, Ohio & 1 & 1350 \\
Milwaukee, WIsconsin & 1 & 1250 \\
San Jose, California & 2 & 1219 \\
Horseheads, New York & 1 & 1200 \\
Kalamazoo, Michigan & 1 & 1130 \\
Warren, Michigan & 1 & 1100 \\
Fort Wayne, Indiana & 2 & 1065 \\
Wausau, Wisconsin & 1 & 680 \\
Minneapolis, Minnesota & 1 & 675 \\
Dayton, Ohio & 3 & 600 \\
Kent, Ohio & 1 & 550 \\
Maple Heights, Ohio & 1 & 512 \\
Evansville, Indiana & 1 & 500 \\
Lebanon, Pennsylvania & 1 & 500 \\
Princeton, Indiana & 1 & 451 \\
Milwaukee, Wisconsin & 3 & 450 \\
Detroit, Michigan & 5 & 350 \\
Manistee, Michigan & 1 & 320 \\
Mattoon, Illinois & 1 & 302 \\
Rock Falls, Illinois & 1 & 262 \\
Coldwater, Michigan & 1 & 217 \\
Springfield, Ohio & 1 & 217 \\
Delphos, Ohio & 1 & 210 \\
Galion, Ohio & 1 & 200 \\
Jeannette, Pennsylvania & 1 & 190 \\
Renton, Washington & 1 & 184 \\
Buffalo, New York & 2 & 150 \\
Cincinnati, Ohio & 1 & 135 \\
Traverse City, Michigan & 1 & 135 \\
Hayward, California & 1 & 134 \\
Lansing, Michigan & 1 & 120 \\
Claremont, California & 1 & 100 \\
Toledo, Ohio & 1 & 80 \\
Wisconsin Dells, Wisconsin & 1 & 80 \\
Anniston, Alabama & 1 & 65 \\
Oak Park, Michigan & 1 & 50 \\
Long Island, New York & 1 & 30 \\
Livonia, Michigan & 1 & 22 \\
Alameda, California & 1 & 3 \\
Elyria, Ohio & 2 & 0 \\
Tulsa, Oklahoma & 2 & 0 \\
Atlanta, Georgia & 1 & 0 \\
Bucyrus, Ohio & 1 & 0 \\
California & 1 & 0 \\
Carrollton, Texas & 1 & 0 \\
Charleston, West Virginia & 1 & 0 \\
Cleveland, Ohio & 1 & 0 \\
Edison, New Jersey & 1 & 0 \\
Elba, Alabama & 1 & 0 \\
Elmira, New York & 1 & 0 \\
Erie, Pennsylvania & 1 & 0 \\
Fostoria, Ohio & 1 & 0 \\
Fullerton, California & 1 & 0 \\
Highland, Illinois & 1 & 0 \\
LaPorte, Indiana & 1 & 0 \\
Louisville, Kentucky & 1 & 0 \\
Lynn, Massachusetts \& Everett, Massachusetts & 1 & 0 \\
Massillon, Ohio & 1 & 0 \\
New Haven, Connecticut & 1 & 0 \\
Oshkosh, Wisconsin & 1 & 0 \\
Pontiac, Michigan & 1 & 0 \\
Port Huron, Michigan & 1 & 0 \\
Roseville, Michigan & 1 & 0 \\
South Bend, Indiana & 1 & 0 \\
St Marys, Ohio & 1 & 0 \\
St Paul, Minnesota & 1 & 0 \\
Troy, New York & 1 & 0 \\
Watertown, New YOrk & 1 & 0 \\
\end{longtable}

\subsubsection{To Do:}\label{to-do}

\begin{itemize}
\tightlist
\item
  continue cleaning up data manually
\item
  clean up cities
\item
  PDF friendly maps
\item
  clean up state column
\end{itemize}

\end{document}
